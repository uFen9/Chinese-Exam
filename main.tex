\documentclass[twocolumn,landscape,UTF8]{ctexart}
\usepackage{lastpage}
\usepackage{color}
\usepackage{wrapfig}%插入图片
\usepackage{tikz}%
\usepackage{harpoon}
\usepackage{tikz}
\usepackage{pgfplots}
\usepackage{tikz,tikz-3dplot}
\usetikzlibrary{shapes.geometric, arrows}
\usepackage{ulem}
\usepackage{titlesec}
\usepackage{graphicx}
\usepackage{colortbl}
\usepackage{amsmath}
\usepackage{listings}
\usepackage{makecell}
\usepackage{indentfirst}
\usepackage{fancyhdr} 
\usepackage{float}
\usepackage{setspace}	% 行间距
\usepackage{bm}    %\boldsymbol 粗体
\usepackage{siunitx}	% 数学
\usepackage{amsmath,amsfonts,amsmath,amssymb,times}
\usepackage{txfonts}
\usepackage{enumerate}	% 编号
\usepackage{tikz,pgfplots}	% 绘图
\usepackage{tkz-euclide,pgfplots}
\usetikzlibrary{automata,positioning}
\usepackage[paperwidth=37cm,paperheight=26cm,top=2cm,bottom=2cm,left=2.7cm]{geometry}
\usepackage{background}
\usepackage[firstpage]{draftwatermark}
\SetWatermarkText{}	% 添加自行编辑的水印
\SetWatermarkLightness{0.85}
\SetWatermarkScale{0.4}
\SetBgContents{}	%去掉draft水印
\lstset{language=C,keywordstyle=\color{red},showstringspaces=false,rulesepcolor=\color{green}}

\textwidth=31.5cm	%文本的宽度 单页

\newsavebox{\zdx}	%装订线

\newcommand{\putzdx}
{
	\marginpar
	{
		\vspace{-1.5cm}
		\rotatebox[origin=c]{90}
		{
  			\usebox{\zdx}
  		}
  	}
}

\newcommand{\me}{\mathrm{e}}  %定义 对数常数e,虚数符号i,j以及微分算子d为直立体。
\newcommand{\mi}{\mathrm{i}}
\newcommand{\mj}{\mathrm{j}}
\newcommand{\dif}{\mathrm{d}}
\newcommand{\bs}{\boldsymbol}%数学黑体
\newcommand{\ds}{\displaystyle}

%选择题
\newcommand{\fourch}[4]{\\\begin{tabular}{*{4}{@{}p{3.5cm}}}A.~#1 & B.~#2 & C.~#3 & D.~#4\end{tabular}} % 四行
\newcommand{\twoch}[4]{\\\begin{tabular}{*{2}{@{}p{7cm}}}A.~#1 & B.~#2\end{tabular}\\\begin{tabular}{*{2}{@{}p{7cm}}}C.~#3 & D.~#4\end{tabular}}  %两行
\newcommand{\onech}[4]{\\A.~#1 \\ B.~#2 \\ C.~#3 \\ D.~#4}  % 一行

\renewcommand{\headrulewidth}{0pt}
\pagestyle{fancy}
\setlength\columnsep{1.5cm}
\begin{document} 

\fancyhf{} %去除默认上下标

%脚注,出现问题调两个参数
\makeatletter
\def\lastfox@putlabel{%
    \immediate\write\@auxout{%
        \string\newlabel{LastFox}{{\@arabic\c@fox}{}{}{}{}}}}
\AtEndDocument{\lastfox@putlabel}
\makeatother
\newcounter{foo}
\newcounter{fox}
\addtocounter{foo}{1}
\addtocounter{fox}{0} 
\fancyfoot[CE,CO]
{
	\hspace*{0.75cm}
	高二数学~~第\,\refstepcounter{fox}\thefoo\refstepcounter{foo}\,页 (共~\ref{LastFox}~页)
	\hspace*{12.25cm}
	高二数学~~第\,\refstepcounter{fox}\thefoo\refstepcounter{foo}\,页 (共~\ref{LastFox}~页)
}

%密封线
\sbox{\zdx}
{
	\parbox{24cm}
	{	
		\centering
		\textbf{学校}\underline{\makebox[28mm][c]{}}
		\textbf{姓名}\underline{\makebox[28mm][c]{}}
		\textbf{班级}\underline{\makebox[28mm][c]{}}
		\textbf{座号}\underline{\makebox[28mm][c]{}}\\
		\vspace{2mm}
		\dotfill{}
	}
}

\setlength{\marginparsep}{1.25cm}%密封线与文档之间的间隔;

%非常重要!!!用于奇数页添加密封线
%-----------------------------------
\putzdx %装订线释放
%-----------------------------------

%标题,局部替换即可
\begin{spacing}{1.25}
	\setstretch{1.5}
	\begin{center}
	{
		%字与字之间必须加上 \quad 全角空格
		\fontsize{17pt}{\baselineskip} \selectfont 题目自拟\\[9pt]		
		\textbf{\fontsize{22pt}{\baselineskip}\selectfont 数\quad 学}
	}
	\end{center}
\end{spacing}

\vspace{10pt}

\textbf{注意事项:}

1. 答题前,考生务必在试题卷、答题卡规定的地方填写自己的准考证号、姓名。

2. 回答选择题时,选出每小题答案后,用铅笔把答题卡上对应题目的答案标号涂黑。如需改动,用橡皮擦干净后,再选涂其他答案标号。回答非选择题时,将答案写在答题卡上。写在本试卷上无效。

3. 考试结束,考生必须将试题卷和答题卡一并交回。

\vspace{6pt}

\begin{spacing}{1.2}

	\noindent\textbf{一、选择题:本题共~8~小题,每小题~5~分,共~40~分。在每小题给出的四个选项中,只有一项是符合题目要求的。}\hangindent=2em
		\begin{enumerate}
		\setcounter{enumi}{0}
	
			\item blank % \onech \twoch \fourch 分别对应每行1、2、4个选项,括号中填写选项内容即可,无需ABCD 

	\end{enumerate}
	
	\noindent\textbf{二、多项选择题:本题共~3~小题,每小题~6~分,共~18~分。在每小题给出的选项中,有多项符合题目要求。全部选对的得~6~分,部分选对的得部分分,有选错的得~0~分。}\hangindent=2em
		\begin{enumerate}
		\setcounter{enumi}{8}

			\item blank

	\end{enumerate}

	\noindent\textbf{三、填空题:本题共~3~小题,每小题~5~分,共~15~分。}\hangindent=2em
		\begin{enumerate}
		\setcounter{enumi}{11}
	
			\item 甲乙两人各有四张卡片,每张卡片上标有一个数字,甲的卡片上分别标有数字$1$,$3$,$5$,$7$,乙的卡片上分别标有数字$2$,$4$,$6$,$8$.两人进行四轮比赛,在每轮比赛中,两人
			各自从自己持有的卡片中随机选一张,并比较所选卡片上的数字大小,数字大的人得$1$分,数字小的人得$0$分,然后各自弃置此轮所选的卡片(弃置的卡片在此后的轮次中不能使用),则四轮比赛后,甲的总得分不小于$2$的概率为
			\underline{\qquad \qquad}.
			%此处举出该题仅为方便CV下划线

	\end{enumerate}
	
	\noindent\textbf{四、解答题:本题共~5~小题,共~77~分。解答应写出文字说明、证明过程或演算步骤。}\hangindent=2em
		\begin{enumerate}
		\setcounter{enumi}{14}

		\item (\,13\,分\,) \\
		blank

		\item (\,15\,分\,) \\
		blank

		\item (\,15\,分\,) \\
		blank

		\item (\,17\,分\,) \\
		blank

		\item (\,17\,分\,) \\
		blank

		\end{enumerate}

\end{spacing}	
\end{document}